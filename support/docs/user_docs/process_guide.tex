% Process Guide (PDF)
\documentclass[11pt]{article}
\usepackage[margin=1in]{geometry}
\usepackage{hyperref}
\usepackage{enumitem}
\setlist{nosep}

\title{Process Guide: Bug Management and Project Management}
\author{Vibe Template Team}
\date{2025-12-22}

\begin{document}
\maketitle

\section*{Purpose}
This guide explains how to use this repository for bug management and project management. It also outlines how to start a new project repo from this template.

\section*{Quick Start}
\begin{enumerate}
  \item Read the process and personas: \path{support/docs/process.md}, \path{support/docs/project_wide_docs/personas.md}.
  \item Review the project plan: \path{support/docs/projectplan.md}.
  \item Start the doc watcher for automatic updates while editing (from the repo root):
  \begin{itemize}
    \item \texttt{python3 support/scripts/watch\_docs.py}
  \end{itemize}
  \item Update issues in \path{support/bugmgmt/issues/issues.jsonl} and regenerate exports:
  \begin{itemize}
    \item \texttt{python3 support/scripts/issues.py list --format json --output} \\
      \texttt{support/bugmgmt/exports/json/bugmgmt\_issues.json}
    \item \texttt{python3 support/scripts/issues.py list --format html --output} \\
      \texttt{support/ui/bugmgmt\_issues.html}
  \end{itemize}
  \item Open \path{support/ui/PM.html} to review project status and key links.
\end{enumerate}

\section*{Repo Layout}
\begin{itemize}
  \item Support tooling lives under \texttt{support/} (process docs, BugMgmt, UI reports).
  \item Project planning docs live under \texttt{projects/}.
  \item Product work goes under \texttt{src/}, \texttt{tests/}, \texttt{config/}, \texttt{data/}, \texttt{assets/}, \texttt{scripts/}, \texttt{docs/}.
\end{itemize}

\section*{Multi-Context Workflow}
\begin{itemize}
  \item Keep three windows open: the project plan, the active phase definition/action plan, and the current stage action file.
  \item Keep personas visible and log decisions in the stage action file as you work.
  \item Use the stage action file as the single place to capture scope, acceptance, and validation notes.
\end{itemize}

\section*{Bug Management Workflow}
\begin{itemize}
  \item Source of truth: \path{support/bugmgmt/issues/issues.jsonl} (one JSON object per line).
  \item Use project-based IDs (example: \texttt{BMG-2025-12-001}).
  \item Required fields:
  \begin{itemize}
    \item \texttt{id, date, project, phase, stage, area, status}
    \item \texttt{severity, summary, owner}
  \end{itemize}
  \item Detail fields: \texttt{root\_cause, proposed\_fix, qa\_reproduction}.
  \item After any bug change, regenerate the JSON and HTML exports.
\end{itemize}

\section*{Project Management Workflow}
\begin{itemize}
  \item Maintain project structure in \texttt{projects/<project>/}.
  \item Track phases in \texttt{phase\_definition.md} and \texttt{action\_plan\_phaseNN.md}.
  \item Log progress in stage action files under \texttt{actions/}.
  \item Update \path{support/ui/PM.html} with current phase status, owners, and next actions.
  \item Render formatted docs for offline viewing:
  \begin{itemize}
    \item \texttt{python3 support/scripts/render\_docs.py}
  \end{itemize}
\end{itemize}

\section*{Starting a New Project Repo}
\begin{enumerate}
  \item Copy this repository to a new location (or clone and rename). Treat the template as read-only.
  \item Optional: run \texttt{python3 support/scripts/init\_project.py --project <project>} \\
    \texttt{--prefix <PREFIX> --owner "Name"}. \\
    This scaffolds docs and updates \path{support/docs/projectplan.md} and \path{support/ui/PM.html}.
  \item If you did not use the script, update \path{support/docs/projectplan.md} and create \\
    \texttt{projects/<project>/}.
  \item Add phase definitions and action plans using templates in \path{support/docs/templates/}.
  \item Set a project prefix in \path{support/scripts/issues.py} (\texttt{PROJECT\_PREFIXES}).
  \item Regenerate BugMgmt exports and open \path{support/ui/PM.html} to verify.
\end{enumerate}

\section*{Notes}
\begin{itemize}
  \item Keep content PII-free unless explicitly required.
  \item Prefer local assets and deterministic outputs.
  \item Update this guide when the process changes.
\end{itemize}

\end{document}
