% Process Guide (PDF)
\documentclass[11pt]{article}
\usepackage[margin=1in]{geometry}
\usepackage{hyperref}
\usepackage{enumitem}
\setlist{nosep}

\newcommand{\docversion}{1.7}
\title{Process Guide: AI-first Project and Bug Management}
\author{Paul Doyle}
\date{2025-12-22 \\ Version \docversion}

\begin{document}
\maketitle

\section*{Overview}
How to use AI\_first for process, project management, and bug management. Use support views for context, personas for decisions, and stage action files to capture evidence.

\section*{Activation (First Session)}
\begin{itemize}
  \item Copy the \texttt{AI\_first/} directory into your repo root.
  \item Open \path{AI_first/ui/index.html} as the bundle landing page.
  \item Open \path{AI_first/ui/PM.html} and \\
    \path{AI_first/ui/bugmgmt_issues.html} to identify active work and open issues.
  \item Choose the active project, bug, or action you are taking on.
  \item Create or update the stage action file at \\
    \texttt{AI\_first/projects/<project>/phases/phase<NN>/actions/} \\
    \texttt{<project>\_phase<NN>\_stage\_<name>\_action.md}.
  \item Record decisions and updates in the stage action file as you go.
\end{itemize}

\subsection*{Context Launch Prompt}
Use this before selecting any persona so every session starts from the same entry point.
Use the Context Launch prompt in \path{AI_first/ui/index.html} or \path{AI_first/ui/process_guide.html}
and copy it into your AI session.
\begin{itemize}
  \item \texttt{Context Launch: Review AI\_first/ui/index.html,} \\
    \texttt{AI\_first/ui/PM.html, AI\_first/ui/bugmgmt\_issues.html,} \\
    \texttt{AI\_first/docs/process.md,} \\
    \texttt{AI\_first/docs/projectplan.md, and AI\_first/docs/project\_wide\_docs/personas.md.} \\
    \texttt{Identify the active project, phase, and stage action file; list open bugs;} \\
    \texttt{summarize current status; and ask which persona to activate next} \\
    \texttt{(default order: Project Creator/Owner -> Project/Process Manager ->} \\
    \texttt{Developer -> QA Lead; add optional personas from} \\
    \texttt{AI\_first/docs/project\_wide\_docs/personas.md only when scope} \\
    \texttt{triggers them).} \\
    \texttt{If no stage action file exists, propose the file path} \\
    \texttt{using the naming convention.}
\end{itemize}

\subsection*{Repo Layout}
\begin{itemize}
  \item Support tooling lives under \texttt{AI\_first/} (process docs, Bug Management (BugMgmt), UI reports).
  \item Project planning docs live under \texttt{AI\_first/projects/}.
  \item Add product workspace directories in downstream repos as needed (outside \texttt{AI\_first/}).
\end{itemize}

\subsection*{Optional Automation}
Use scripts only when you want to regenerate UI views. The day-to-day workflow is AI-first and doc-driven.
\begin{itemize}
  \item \texttt{python3 AI\_first/scripts/render\_docs.py} to render markdown into \texttt{AI\_first/ui/docs/}.
  \item \texttt{python3 AI\_first/scripts/watch\_docs.py} to auto-render while editing.
  \item \texttt{python3 AI\_first/scripts/init\_project.py --project <project>} \\
    \texttt{--prefix <PREFIX> --owner "Name"} to scaffold a project.
  \item \texttt{python3 AI\_first/scripts/issues.py list --format json --output} \\
    \texttt{AI\_first/bugmgmt/exports/json/bugmgmt\_issues.json} to regenerate JSON.
  \item \texttt{python3 AI\_first/scripts/issues.py list --format html --output} \\
    \texttt{AI\_first/ui/bugmgmt\_issues.html} to regenerate the HTML report.
\end{itemize}

\subsection*{Notes}
\begin{itemize}
  \item Keep content PII-free unless explicitly required.
  \item Prefer local assets and deterministic outputs.
  \item Document versioning: 1.\textit{x}; increment the minor number for each change to this guide.
  \item Bug Management (BugMgmt) is optional; remove or ignore \texttt{AI\_first/bugmgmt/}, \\
    \texttt{AI\_first/ui/bugmgmt\_issues.html}, and \texttt{AI\_first/scripts/issues.py} \\
    if you do not need it, and remove the Bug Management link from navigation if desired.
  \item Run scripts from the repo root; use \texttt{python3} (or \texttt{python} if mapped to Python 3).
  \item Update this guide when the process changes.
\end{itemize}

\section*{AI-first Loop}
\begin{itemize}
  \item Review Context: ask Codex to review the AI\_first context before work begins. \\
    Output: current phase and active stage action file summary.
  \item Activate Personas: use personas to define actions, acceptance criteria, and risks before implementation. \\
    Output: decisions recorded in the stage action file.
  \item Plan and Update: update the stage action file and phase docs to reflect the next actions. \\
    Output: updated phase definition/action plan and stage action log.
  \item Execute: execute actions, validate outcomes, and record results. \\
    Output: validation notes and completion status logged.
\end{itemize}

\subsection*{AI Prompt Examples}
\begin{itemize}
  \item Review Context: \texttt{Context Launch: Review AI\_first/ui/index.html,} \\
    \texttt{AI\_first/ui/PM.html, AI\_first/ui/bugmgmt\_issues.html,} \\
    \texttt{AI\_first/docs/process.md,} \\
    \texttt{AI\_first/docs/projectplan.md, and AI\_first/docs/project\_wide\_docs/personas.md.} \\
    \texttt{Identify the active project, phase, and stage action file; list open bugs;} \\
    \texttt{summarize current status; and ask which persona to activate next} \\
    \texttt{(default order: Project Creator/Owner -> Project/Process Manager ->} \\
    \texttt{Developer -> QA Lead; add optional personas from} \\
    \texttt{AI\_first/docs/project\_wide\_docs/personas.md only when scope} \\
    \texttt{triggers them).} \\
    \texttt{If no stage action file exists, propose the file path} \\
    \texttt{using the naming convention.}
  \item Activate Personas: \texttt{Activate Project/Process Manager Persona. Review the active} \\
    \texttt{stage action file and propose 3 next actions with acceptance criteria} \\
    \texttt{and risks.}
  \item Plan and Update: \texttt{Draft updates for the stage action file and related project} \\
    \texttt{docs. Note next steps and owners.}
  \item Execute: \texttt{Act as Developer Persona. Implement the chosen action, summarize} \\
    \texttt{changes, and draft closure notes for the stage action file} \\
    \texttt{(and Bug Management entry if applicable).}
\end{itemize}

\subsection*{Artifact Map (Source of Truth)}
This map mirrors the AI-first Loop and points to the source-of-truth artifacts for each step.
\begin{itemize}
  \item Review Context: \path{AI_first/ui/PM.html}, \path{AI_first/ui/bugmgmt_issues.html}, \\
    \path{AI_first/docs/process.md}, \path{AI_first/docs/projectplan.md}.
  \item Activate Personas: \path{AI_first/docs/project_wide_docs/personas.md}, \\
    \path{AI_first/docs/process.md}. Record decisions in the stage action file.
  \item Plan and Update: \path{AI_first/docs/templates/phase_definition_template.md}, \\
    \path{AI_first/docs/templates/phase_action_plan_template.md}, \\
    \path{AI_first/docs/templates/action_stage_template.md}.
  \item Execute: \path{AI_first/ui/PM.html} (and Bug Management when work is bug-driven). \\
    Log validation notes in the stage action file.
\end{itemize}

\section*{Personas and Windows}
Default to three terminals or windows: Project Creator/Owner (switching to Project/Process Manager once the project is defined), Developer, and QA. Keep each perspective visible while AI drafts and you validate the work.
\begin{itemize}
  \item Terminal 1 (Project Creator/Owner then Project/Process Manager): define the project setup, then review the project plan, active phase docs, and action plan in one view; log decisions and keep project actions aligned.
  \item Terminal 2 (Developer): work on implementation, track local changes, and validate results; update stage action files as work completes.
  \item Terminal 3 (QA): draft and verify bug entries, confirm reproduction steps, and log test notes; use Bug Management to track open and closed issues.
\end{itemize}

\subsection*{AI Prompt Examples}
\begin{itemize}
  \item \textbf{Project Creator/Owner Persona}: \\
    \texttt{Activate Project Creator/Owner Persona. Confirm project name,} \\
    \texttt{purpose, owner, prefix, and initial phases to create.}
  \item \textbf{Project/Process Manager Persona}: \\
    \texttt{Activate Project/Process Manager Persona. Select the active} \\
    \texttt{project and phase, review the stage action file, and continue from the last} \\
    \texttt{update with 3 next actions and acceptance criteria.}
  \item \textbf{Developer Persona}: \\
    \texttt{Activate Developer Persona. Select the active project and} \\
    \texttt{phase, continue from the last update, and carry out the actions already} \\
    \texttt{identified or fix a logged bug. Summarize changes and draft updates for} \\
    \texttt{the stage action file and bug entry.}
  \item \textbf{QA Lead Persona}: \\
    \texttt{Activate QA Lead Persona. Select the active project and phase,} \\
    \texttt{continue from the last update, and either log a bug or outline a test} \\
    \texttt{automation suite. Draft validation notes for the stage action file or} \\
    \texttt{bug entry.}
\end{itemize}

\subsection*{Persona Responsibilities}
\begin{itemize}
  \item \textbf{Project Creator/Owner Persona}: establish project name, purpose, owner, and prefix; define initial phases and files to create.
  \item \textbf{Project/Process Manager Persona}: \\
    define phases and actions in \path{AI_first/docs/projectplan.md} and phase docs; confirm scope and acceptance criteria; use \path{AI_first/ui/PM.html} as the portfolio view.
  \item \textbf{Developer Persona}: execute the active stage action file; analyze root cause and propose fixes in Bug Management; update status as work moves to \texttt{in\_progress} and \texttt{closed}.
  \item \textbf{QA Lead Persona}: draft bug entries with AI and confirm reproduction steps; validate fixes and update notes for closed issues; use \path{AI_first/ui/bugmgmt_issues.html} for audit visibility.
\end{itemize}

\subsection*{Optional Personas (Scope-triggered)}
Use optional personas only when the scope or risk warrants specialized review. See \path{AI_first/docs/project_wide_docs/personas.md} for full prompts.
\begin{itemize}
  \item \textbf{Product Manager}: define acceptance criteria and scope boundaries.
  \item \textbf{Repository Steward}: protect the template structure and AI\_first/project separation.
  \item \textbf{Docs Expert}: keep docs accurate, consistent, and linked.
  \item \textbf{UI/Accessibility}: review usability, accessibility, and style consistency.
  \item \textbf{Bug Triage}: validate IDs, required fields, and owners after bug updates.
  \item \textbf{Automation/Tooling}: list commands needed to refresh exports and rendered docs.
  \item \textbf{Architect}: identify cross-boundary risks and acceptance conditions.
  \item \textbf{Security}: flag threats, secrets/PII handling, and mitigations.
  \item \textbf{Ops/Observability}: note runbook gaps, monitoring needs, and rollback notes.
  \item \textbf{Performance/Cost}: note performance or cost risks when scale matters.
  \item \textbf{DBA}: note data integrity and schema risks when a database is introduced.
\end{itemize}

\section*{Project Planning \& Delivery}
Keep project planning AI-first: define phase intent, plan actions, then deliver and review outcomes while PM.html mirrors progress.
Use the appropriate persona to draft, execute, and validate updates.
Primary artifacts: \texttt{phase\_definition.md}, \texttt{action\_plan\_phaseNN.md}, the active stage action file, and \path{AI_first/ui/PM.html}.
\begin{itemize}
  \item Define Phase: update phase goals, scope, and acceptance criteria. Output: refreshed phase definition.
  \item Plan Actions: break work into actions in the action plan; create or update the active stage action file. Output: actionable plan and stage log.
  \item Deliver \& Review: execute actions, validate outcomes against acceptance criteria, and update PM.html and next actions.
\end{itemize}

\subsection*{AI Prompt Examples}
\begin{itemize}
  \item Define Phase: \texttt{Select the active project and phase, continue from the last} \\
    \texttt{update, and draft updates to the phase definition with goals, scope,} \\
    \texttt{and acceptance criteria.}
  \item Plan Actions: \texttt{Using the current phase definition, draft or update the action} \\
    \texttt{plan and the active stage action file with next actions, owners,} \\
    \texttt{and checkpoints.}
  \item Deliver \& Review: \texttt{Summarize delivery progress and validation results,} \\
    \texttt{update the stage action file with outcomes, and list the next actions} \\
    \texttt{to reflect in PM.html.}
\end{itemize}

\section*{Bug Management Workflow}
Keep bugs AI-first: capture the issue, work the fix, and confirm closure with evidence.
Use the appropriate persona to draft, execute, and validate updates.
Primary artifacts: \texttt{issues.jsonl}, the active stage action file, and \path{AI_first/ui/bugmgmt_issues.html}.
\begin{itemize}
  \item Open: draft a clear bug report with reproduction steps and impact; confirm summary and severity before logging. Output: new entry in \texttt{issues.jsonl}.
  \item Work: analyze root cause and propose a fix plan; track progress and decisions in the stage action file. Output: status updated as work moves forward.
  \item Close: record closure notes and validation results; reflect outcomes in project actions if needed. Output: bug closed with evidence recorded.
\end{itemize}

\subsection*{AI Prompt Examples}
\begin{itemize}
  \item Open: \texttt{Draft a bug entry with required fields, clear reproduction steps, and} \\
    \texttt{impact notes. Suggest severity and summary for review.}
  \item Work: \texttt{Given this bug entry, analyze root cause, propose a fix plan, and} \\
    \texttt{draft updates for the stage action file and bug status.}
  \item Close: \texttt{Summarize fix outcomes, draft closure notes, and list any project plan} \\
    \texttt{updates needed.}
\end{itemize}

\end{document}
