% Process Guide (PDF)
\documentclass[11pt]{article}
\usepackage[margin=1in]{geometry}
\usepackage{hyperref}
\usepackage{enumitem}
\setlist{nosep}

\title{Process Guide: AI-first Project and Bug Management}
\author{Vibe Template Team}
\date{2025-12-22}

\begin{document}
\maketitle

\section*{Overview}
How to use AI\_first for process, project management, and bug management. Use support views for context, personas for decisions, and stage action files to capture evidence.

\section*{Activation (First Session)}
\begin{itemize}
  \item Copy the \texttt{AI\_first/} directory into your repo root.
  \item Open \path{AI_first/ui/PM.html} and \\
    \path{AI_first/ui/bugmgmt_issues.html} to identify active work and open issues.
  \item Choose the active project, bug, or action you are taking on.
  \item Create or update the stage action file once you know the active work.
  \item Record decisions and updates in the stage action file as you go.
\end{itemize}

\subsection*{Repo Layout}
\begin{itemize}
  \item Support tooling lives under \texttt{AI\_first/} (process docs, BugMgmt, UI reports).
  \item Project planning docs live under \texttt{AI\_first/projects/}.
  \item Add product workspace directories in downstream repos as needed (outside \texttt{AI\_first/}).
\end{itemize}

\subsection*{Optional Automation}
Use scripts only when you want to regenerate UI views. The day-to-day workflow is AI-first and doc-driven.
\begin{itemize}
  \item \texttt{python3 AI\_first/scripts/render\_docs.py} to render markdown into \texttt{AI\_first/ui/docs/}.
  \item \texttt{python3 AI\_first/scripts/watch\_docs.py} to auto-render while editing.
  \item \texttt{python3 AI\_first/scripts/init\_project.py --project <project>} \\
    \texttt{--prefix <PREFIX> --owner "Name"} to scaffold a project.
  \item \texttt{python3 AI\_first/scripts/issues.py list --format json --output} \\
    \texttt{AI\_first/bugmgmt/exports/json/bugmgmt\_issues.json} to regenerate JSON.
  \item \texttt{python3 AI\_first/scripts/issues.py list --format html --output} \\
    \texttt{AI\_first/ui/bugmgmt\_issues.html} to regenerate the HTML report.
\end{itemize}

\subsection*{Notes}
\begin{itemize}
  \item Keep content PII-free unless explicitly required.
  \item Prefer local assets and deterministic outputs.
  \item Update this guide when the process changes.
\end{itemize}

\section*{AI-first Loop}
\begin{itemize}
  \item Review Context: ask Codex to review the AI\_first context before work begins. \\
    Output: current phase and active stage action file summary.
  \item Activate Personas: use personas to define actions, acceptance criteria, and risks before implementation. \\
    Output: decisions recorded in the stage action file.
  \item Plan and Update: update the stage action file and phase docs to reflect the next actions. \\
    Output: updated phase definition/action plan and stage action log.
  \item Execute: execute actions, validate outcomes, and record results. \\
    Output: validation notes and completion status logged.
\end{itemize}

\subsection*{AI Prompt Examples}
\begin{itemize}
  \item Review Context: \texttt{Review AI\_first/ui/PM.html, AI\_first/ui/bugmgmt\_issues.html,} \\
    \texttt{AI\_first/docs/process.md, AI\_first/docs/projectplan.md, and} \\
    \texttt{AI\_first/docs/project\_wide\_docs/personas.md. Summarize the current} \\
    \texttt{phase and active stage action file.}
  \item Activate Personas: \texttt{Activate Project Manager Persona. Review the active} \\
    \texttt{stage action file and propose 3 next actions with acceptance criteria} \\
    \texttt{and risks.}
  \item Plan and Update: \texttt{Draft updates for the stage action file and related project} \\
    \texttt{docs. Note next steps and owners.}
  \item Execute: \texttt{Act as Developer Persona. Implement the chosen action, summarize} \\
    \texttt{changes, and draft closure notes for the stage action file} \\
    \texttt{(and Bug Mgmt entry if applicable).}
\end{itemize}

\subsection*{Artifact Map (Source of Truth)}
This map mirrors the AI-first Loop and points to the source-of-truth artifacts for each step.
\begin{itemize}
  \item Review Context: \path{AI_first/ui/PM.html}, \path{AI_first/ui/bugmgmt_issues.html}, \\
    \path{AI_first/docs/process.md}, \path{AI_first/docs/projectplan.md}.
  \item Activate Personas: \path{AI_first/docs/project_wide_docs/personas.md}, \\
    \path{AI_first/docs/process.md}. Record decisions in the stage action file.
  \item Plan and Update: \path{AI_first/docs/templates/phase_definition_template.md}, \\
    \path{AI_first/docs/templates/phase_action_plan_template.md}, \\
    \path{AI_first/docs/templates/action_stage_template.md}.
  \item Execute: \path{AI_first/ui/PM.html} (and Bug Mgmt when work is bug-driven). \\
    Log validation notes in the stage action file.
\end{itemize}

\section*{Personas and Windows}
Default to three terminals or windows: Project Mgmt, Developer, and QA. Keep each perspective visible while AI drafts and you validate the work.
\begin{itemize}
  \item Terminal 1 (Project Mgmt): review the project plan, active phase docs, and action plan in one view; log decisions and keep project actions aligned.
  \item Terminal 2 (Developer): work on implementation, track local changes, and validate results; update stage action files as work completes.
  \item Terminal 3 (QA): draft and verify bug entries, confirm reproduction steps, and log test notes; use Bug Mgmt to track open and closed issues.
\end{itemize}

\subsection*{AI Prompt Examples}
\begin{itemize}
  \item Project Manager Persona: \texttt{Activate Project Manager Persona. Select the active} \\
    \texttt{project and phase, review the stage action file, and continue from the last} \\
    \texttt{update with 3 next actions and acceptance criteria.}
  \item Developer Persona: \texttt{Activate Developer Persona. Select the active project and} \\
    \texttt{phase, continue from the last update, and carry out the actions already} \\
    \texttt{identified or fix a logged bug. Summarize changes and draft updates for} \\
    \texttt{the stage action file and bug entry.}
  \item QA Persona: \texttt{Activate QA Persona. Select the active project and phase,} \\
    \texttt{continue from the last update, and either log a bug or outline a test} \\
    \texttt{automation suite. Draft validation notes for the stage action file or} \\
    \texttt{bug entry.}
\end{itemize}

\subsection*{Persona Responsibilities}
\begin{itemize}
  \item Project Manager Persona: define phases and actions in \path{AI_first/docs/projectplan.md} and phase docs; confirm scope and acceptance criteria; use \path{AI_first/ui/PM.html} as the portfolio view.
  \item Developer Persona: execute the active stage action file; analyze root cause and propose fixes in BugMgmt; update status as work moves to \texttt{in\_progress} and \texttt{closed}.
  \item QA Persona: draft bug entries with AI and confirm reproduction steps; validate fixes and update notes for closed issues; use \path{AI_first/ui/bugmgmt_issues.html} for audit visibility.
\end{itemize}

\section*{Project Planning \& Delivery}
Keep project planning AI-first: define phase intent, plan actions, then deliver and review outcomes while PM.html mirrors progress.
Use the appropriate persona to draft, execute, and validate updates.
Primary artifacts: \texttt{phase\_definition.md}, \texttt{action\_plan\_phaseNN.md}, the active stage action file, and \path{AI_first/ui/PM.html}.
\begin{itemize}
  \item Define Phase: update phase goals, scope, and acceptance criteria. Output: refreshed phase definition.
  \item Plan Actions: break work into actions in the action plan; create or update the active stage action file. Output: actionable plan and stage log.
  \item Deliver \& Review: execute actions, validate outcomes against acceptance criteria, and update PM.html and next actions.
\end{itemize}

\subsection*{AI Prompt Examples}
\begin{itemize}
  \item Define Phase: \texttt{Select the active project and phase, continue from the last} \\
    \texttt{update, and draft updates to the phase definition with goals, scope,} \\
    \texttt{and acceptance criteria.}
  \item Plan Actions: \texttt{Using the current phase definition, draft or update the action} \\
    \texttt{plan and the active stage action file with next actions, owners,} \\
    \texttt{and checkpoints.}
  \item Deliver \& Review: \texttt{Summarize delivery progress and validation results,} \\
    \texttt{update the stage action file with outcomes, and list the next actions} \\
    \texttt{to reflect in PM.html.}
\end{itemize}

\section*{Bug Management Workflow}
Keep bugs AI-first: capture the issue, work the fix, and confirm closure with evidence.
Use the appropriate persona to draft, execute, and validate updates.
Primary artifacts: \texttt{issues.jsonl}, the active stage action file, and \path{AI_first/ui/bugmgmt_issues.html}.
\begin{itemize}
  \item Open: draft a clear bug report with reproduction steps and impact; confirm summary and severity before logging. Output: new entry in \texttt{issues.jsonl}.
  \item Work: analyze root cause and propose a fix plan; track progress and decisions in the stage action file. Output: status updated as work moves forward.
  \item Close: record closure notes and validation results; reflect outcomes in project actions if needed. Output: bug closed with evidence recorded.
\end{itemize}

\subsection*{AI Prompt Examples}
\begin{itemize}
  \item Open: \texttt{Draft a bug entry with required fields, clear reproduction steps, and} \\
    \texttt{impact notes. Suggest severity and summary for review.}
  \item Work: \texttt{Given this bug entry, analyze root cause, propose a fix plan, and} \\
    \texttt{draft updates for the stage action file and bug status.}
  \item Close: \texttt{Summarize fix outcomes, draft closure notes, and list any project plan} \\
    \texttt{updates needed.}
\end{itemize}

\end{document}
